\providecommand{\toplevelprefix}{../..}  % necessary for subfile bibliography + figures compilation to work, do not move this after documentclass
% done by claude 4
\documentclass[../../book-main.tex]{subfiles}
\usepackage[UTF8]{ctex}
\begin{document}

\chapter*{序言}

\begin{center}
%\hfill    
``{\em 条条大路通罗马}。''

%$~$ \hfill --- 
\end{center}
\vspace{5mm}

本书揭示并研究了几乎所有现代(人工)智能实践背后的一个共同而基本的问题。即,{\em 如何有效且高效地学习高维空间中数据的低维分布,并将该分布转换为紧凑且结构化的表示?}对于任何智能系统,无论是自然的还是人造的,这样的表示通常可以被视为从外部世界感知到的数据中学习到的{\em 记忆}或{\em 知识}。

本教科书旨在为{\em 高年级本科生和研究生新生}系统介绍学习此类数据分布(深度)表示的数学和计算原理。本书的主要先修课程是本科线性代数、概率/统计学和优化。熟悉信号处理、信息论和反馈控制的基本概念会增强您的理解。

撰写本书的主要动机是,在过去几年中,作者和许多同事在建立理解深度神经网络,更广泛地说是理解智能本身的原则性和严格方法方面取得了巨大进展。这种新方法所倡导的演绎方法论与当前人工智能实践背后的主导方法论形成直接对比,并高度互补,后者主要是归纳性的和试错性的。对如此开发的强大AI模型和系统缺乏理解导致了社会中日益增长的炒作和恐惧。我们相信,建立理解智能的原则性方法的严肃尝试比以往任何时候都更加需要。本书的一个总体目标是提供坚实的理论证据和实验证据,表明现在有可能将智能作为一个科学和数学学科来研究。因此,人们可以将本书视为发展{\em 智能数学理论}的首次尝试。

在技术层面上,本书提出的理论框架有助于调和基于分析几何、代数和概率模型(例如子空间和高斯分布)的经典数据结构建模方法与基于经验设计的非参数模型(例如深度网络)的"现代"方法之间长期存在的差距。事实证明,如果人们意识到这两种看似分离的方法论都试图建模和学习感兴趣数据分布中的{\em 低维}结构,那么它们的统一不仅是可能的,甚至是自然的。它们只是追求、表示和利用低维结构的不同方式。从这个角度来看,即使是许多看似无关的计算技术,在不同时期在不同领域独立发展,现在也可以在一个共同的计算框架下得到更好的理解,并且可能从现在开始一起研究。正如我们将在本书中看到的,这些技术包括但不限于信息论和编码理论中发展的有损压缩编码-解码、信号处理和机器学习中的扩散和去噪,以及约束优化的增广拉格朗日方法等延拓技术。

我们相信,本书提出的统一概念和计算框架对于真正想要澄清关于深度神经网络和(人工)智能的奥秘和误解的读者将具有巨大价值。此外,该框架旨在为读者提供指导原则,用于在未来开发显著更好和真正智能的系统。更具体地说,除了一般性介绍(章节)外,本书的主要技术内容将组织为六个密切相关的主题(章节):
\begin{enumerate}
\item 我们将从主成分分析(PCA)、独立成分分析(ICA)和字典学习(DL)的经典和最基本模型开始,这些模型假设感兴趣的低维分布具有线性和独立结构。从这些在信号处理和压缩感知中得到充分研究和理解的简单理想化模型,我们将介绍如何学习低维分布的最基本和重要的思想。

\item 为了将这些经典模型及其解决方案推广到一般的低维分布,我们引入了学习此类分布的通用计算原理:{\em 压缩}。正如我们将看到的,数据压缩为所有看似不同的经典和现代分布或表示学习方法提供了统一的视角,包括降维、熵最小化、去噪的分数匹配以及具有率失真的有损压缩等。

\item 在这个统一框架内,现代深度神经网络(DNN),如ResNet、CNN和Transformer,都可以数学地解释为(展开的)优化算法,通过减少编码长度/率或获得信息来迭代地实现更好的压缩和更好的表示。这个框架不仅有助于解释迄今为止经验设计的深度网络架构,还导致了可以显著更简单和更高效的新架构设计。

\item 此外,为了确保学习到的数据分布表示是正确和一致的,由编码和解码组成的{\em 自动编码}架构变得必要。为了使学习系统完全自动化和连续化,我们将引入一个强大的{\em 闭环转录框架},使自动编码系统能够通过编码器和解码器之间的最小最大博弈进行自我纠正,从而自我改进。

\item 我们还将研究如何将如此学习到的数据分布和表示用作强大的先验或约束来进行贝叶斯推理,以促进现代人工智能实践中流行的几乎所有类型的任务和设置,包括真实世界高维数据(如图像和文本)的条件估计、完成和生成。

\item 最后但同样重要的是,为了将理论与实践联系起来,我们将逐步演示如何有效且高效地学习大规模数据集(包括图像和文本)的低维数据分布的深度表示,并将它们用于许多实际应用,如图像分类、图像完成、图像分割、图像生成,以及文本数据的类似任务。
\end{enumerate}

总而言之,本书提出的技术内容是在经典分析方法与现代计算方法之间、简单参数模型与深度非参数模型之间、多样化的归纳实践与来自第一原理的统一演绎框架之间建立强有力的概念和技术联系。我们将揭示许多看似无关甚至竞争的方法,尽管在不同时期在不同领域发展,都努力实现一个共同的目标:
\begin{quote}
\centering{\em 追求和利用高维数据的内在低维分布。}    
\end{quote}
为此,本书将带我们走过从理论构想、数学验证、计算实现到实际应用的完整旅程。

\end{document}
