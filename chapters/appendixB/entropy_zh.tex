
\providecommand{\toplevelprefix}{../..}  % necessary for subfile bibliography + figures compilation to work, do not move this after documentclass
\documentclass[../../book-main.tex]{subfiles}
\usepackage[UTF8]{ctex}
\begin{document}

\chapter{熵、扩散与去噪}\label{app:entropy}\label{app:diffusion-denoising}

\begin{quote}
``{\em 无序度或熵随时间的增加,是所谓时间之矢的一个例子,它区分了过去与未来,赋予了时间一个方向。}''

$~$\hfill -- 《时间简史》,史蒂芬·霍金
 \end{quote}
\vspace{5mm}

在本附录中,我们将为\Cref{ch:general-distribution}中提到的几个事实提供证明,这些事实与微分熵以及它在扩散过程下的演化有关。我们将对代表数据源的随机变量(我们称之为\(\vx\))做出以下温和的假设。

\begin{assumption}\label{assumption:entropy_x_compact_support}
    \(\vx\)的支撑集为一个紧集\(\cS \subseteq \R^{D}\),其半径至多为\(R\),即\(R := \sup_{\vxi \in \cS}\norm{\vxi}_{2}\)。
\end{assumption}

特别地,由于欧几里得空间中的紧集是有界的,因此有\(R < \infty\)。我们将统一使用记号\(B_{r}(\vxi) := \{\vu \in \R^{D} \colon \norm{\vxi - \vu}_{2} \leq r\}\)来表示以\(\vxi\)为中心、半径为\(r\)的欧几里得球。在这个意义上,\Cref{assumption:entropy_x_compact_support}意味着\(\cS \subseteq B_{R}(\vzero)\)。

请注意,这个假设在实践中对我们关心的(几乎)所有变量都成立,因为它(通常)是在数据预处理过程中的归一化步骤所施加的。

\section{低维分布的微分熵}\label{sec:low_dim_entropy}

在这个简短的附录中,我们为以下事实提供一个证明。

\begin{theorem}\label{thm:degenerate_entropy_negative_infinity}
    设\(\vx\)为任意满足\Cref{assumption:entropy_x_compact_support}的随机变量,且\(\vx\)的支撑集\(\cS\)的体积为\(0\)。\footnote{严格来说,这意味着\(\cS\)是波莱尔可测的,且其波莱尔测度为\(0\)。} 那么\(h(\vx) = -\infty\)。
\end{theorem}
\begin{proof}
    我们将针对\(\vx\)在\(\cS\)上均匀分布的情况来证明这一点;这种情况抓住了核心思想,而又不过于技术化。其基本思想是对\(\cS\)进行\(\eps\)-增厚,得到\(\cS_{\eps}\),定义为
    \begin{equation}
        S_{\eps} = \bigcup_{\vxi \in \cS}B_{\eps}(\vxi)
    \end{equation}
    如\Cref{fig:entropy_eps_thickening}所示。
    \begin{figure}[th]
        \centering
        \begin{tikzpicture}
            \def\radius{0.2cm} % 定义 epsilon 值
            \def\curve{(0,0) .. controls (1,1.5) and (3,-0.5) .. (4,1)} % 定义曲线 S

            % 沿曲线绘制许多重叠的圆来表示增厚后的 S_eps
            % 使用 decoration 沿路径放置标记(圆)
            \path[decoration={markings, mark=between positions 0 and 1 step 0.02 with {
                    % 在每个标记处填充一个圆;低不透明度以显示重叠
                    \fill[red!30, opacity=0.5, draw=none] (0,0) circle (\radius);
                }}, postaction={decorate}] \curve;

            % 在顶部绘制原始曲线 S,更粗,蓝色
            \draw[blue, thick] \curve node[right, blue] {$\cS$};

            % 在曲线 S 上选择一个示例点 x
            \coordinate (p_on_curve) at (2, 0.5); % 曲线上的一个近似点
            % 为这个 x 绘制特定的球 B_eps(x),颜色稍深/更不透明
            \fill[red!50, opacity=0.7] (p_on_curve) circle (\radius);
            % 标记点 x
            \fill[black] (p_on_curve) circle (1pt) node[below left] {$x$};
            % 添加一个箭头指示这个特定球的半径 epsilon

            % 标记增厚的区域 S_eps,并适当地放置标签
             \node[red, below right] at (3.5, -0.2) {$\cS_{\eps} = \bigcup_{\vxi \in S} B_{\eps}(\vxi)$};

        \end{tikzpicture}
        \caption{曲线\(\cS \subseteq \R^{2}\)的\(\eps\)-增厚\(\cS_\eps\)示意图。}
        \label{fig:entropy_eps_thickening}
    \end{figure}
    我们将处理支撑集为\(\cS_{\eps}\)的随机变量,它是全维的,然后取\(\eps \to 0\)的极限。

    因此,设\(\vx \sim \dUnif(\cS)\),它没有密度\footnote{(相对于\(\R^{D}\)上的波莱尔测度)。},且\(\vx_{\eps} \sim \dUnif(\cS_{\eps})\)。由于\(\cS_{\eps}\)具有正体积,\(\vx_{\eps}\)有一个等于下式的密度\(p_{\eps}\)
    \begin{equation}
        p_{\eps}(\vxi) = \indvar(\vxi \in \cS_{\eps}) \cdot \frac{1}{\volume(\cS_{\eps})}.
    \end{equation}
    由于\(\vx\)没有密度,理解\(h(\vx)\)的方式是通过等式
    \begin{equation}
        h(\vx) = \lim_{\eps \searrow 0}h(\vx_{\eps});
    \end{equation}
    我们将证明后者的极限值为\(-\infty\)。

    确实,根据\(0 \log 0 = 0\)的约定,我们有
    \begin{align}
        h(\vx_{\eps}) 
        &= -\int_{\R^{D}}p_{\eps}(x)\log p_{\eps}(x)\odif{x} \\ 
        &= -\int_{\cS_{\eps}}\frac{1}{\volume(\cS_{\eps})} \log\rp{\frac{1}{\volume(\cS_{\eps})}}\odif{\vxi} \\ 
        &= \frac{\log(\volume(\cS_{\eps}))}{\volume(\cS_{\eps})}\int_{\cS_{\eps}}\odif{\vxi} \\ 
        &= \log(\volume(\cS_{\eps})).
    \end{align}
    由于\(\cS\)是紧集,\(\volume(\cS_{\eps})\)是有限的,并且当\(\eps \searrow 0\)时趋向于\(0\)。\footnote{在\(\cS\)非紧且\(\vx\)在\(\cS\)上非均匀分布的情况下,我们定义一个更精细的\(\vx_{\eps}\)版本,其在\(\vxi\)处的密度在垂直于\(\cS\)和与\(\cS\)对齐的方向上以不同速率变化。这种密度会导致不同的计算,不涉及\(\cS_{\eps}\)的体积,而是涉及一个类似于\(\vx_{\eps}\)所占的有效体积,该体积是有限的,并且当\(\eps \searrow 0\)时趋于零。} 于是
    \begin{equation}
        h(\vx) = \lim_{\eps \searrow 0}h(\vx_{\eps}) = \lim_{\eps \searrow 0}\log(\volume(\cS_{\eps})) = -\infty,
    \end{equation}
    即证。
\end{proof}

上述定理实际上是一组更为著名和重要的结果的推论,这些结果是关于在\(\vx\)的分布满足某些约束条件下,\(\vx\)可能达到的最大熵。我们若不在此处提供这些结果将是一种疏忽,但我们不给出证明;一个合适的参考文献是\cite{poliyanski2024information}。
\begin{theorem}\label{thmx:max_entropy}
    设\(\vx\)是\(\R^{D}\)上的一个随机变量。
    \begin{enumerate}
        \item 如果\(\vx\)的支撑集是一个紧集\(\cS \subseteq \R^{D}\)(即满足\Cref{assumption:entropy_x_compact_support}),那么
        \begin{equation}
            h(\vx) \leq h(\dUnif(\cS)) = \log \volume(\cS).
        \end{equation}
        \item 如果\(\vx\)具有有限协方差,使得对于一个PSD矩阵\(\vSigma \in \PSD(D)\),有\(\Cov(\vx) \preceq \vSigma\)(相对于PSD序,即\(\vSigma - \Cov(\vx)\)是PSD),那么
        \begin{equation}
            h(\vx) \leq h(\dNorm(\vzero, \vSigma)) = \frac{1}{2}\log((2\pi e)^{D}\det\vSigma).
        \end{equation}
        \item 如果\(\vx\)具有有限二阶矩,使得对于一个常数\(a \geq 0\),有\(\Ex\norm{\vx}_{2}^{2} \leq a\),那么
        \begin{equation}
            h(\vx) \leq h\rp{\dNorm\rp{\vzero, \frac{a}{D}\vI}} = \frac{D}{2}\log\frac{2\pi e a}{D}.
        \end{equation}
    \end{enumerate}
\end{theorem}
显然,\Cref{thm:degenerate_entropy_negative_infinity}是\Cref{thmx:max_entropy}.1在\(\volume(\cS) = 0\)时的特例。

\section{扩散与去噪过程}\label{sec:entropy_diffusion}

在正文(\Cref{ch:general-distribution})中,我们考虑了一个随机变量\(\vx\)和一个由\eqref{eq:additive_gaussian_noise_model}定义的随机过程,即:
\begin{equation}\label{eq:app_additive_gaussian_noise_model}
    \vx_{t} = \vx + t\vg, \qquad  \forall t \in [0, T]
\end{equation}
其中\(\vg \sim \dNorm(\vzero, \vI)\)与\(\vx\)独立。

本节的结构如下。在\Cref{sub:diffusion_entropy_increases}中,我们提供一个形式化的定理和清晰的证明,表明在\Cref{eq:app_additive_gaussian_noise_model}下熵是增加的,即\(\odv*{h(\vx_{t})}{t} > 0\)。在\Cref{sub:denoising_entropy_decreases}中,我们提供一个形式化的定理和清晰的证明,表明在\Cref{eq:app_additive_gaussian_noise_model}下,去噪过程中熵是减少的,即对于所有\(s < t\),有\(h(\Ex[\vx_{s} \given \vx_{t}]) < h(\vx_{t})\)。在\Cref{sub:app_diffusion_intermediate_results}中,我们为建立前几小节结论所需的技术性引理提供证明。

在开始之前,我们介绍一些关键的记号。首先,令\(\phi_{t}\)为\(\dNorm(\vzero, t^{2}\vI)\)的密度,即:
\begin{equation}\label{eq:gaussian_noise_time_t}
    \phi_{t}(\vxi) := \frac{1}{(2\pi)^{D/2}t^{D}}\exp\rp{-\frac{\norm{\vxi}_{2}^{2}}{t^{2}}}.
\end{equation}
其次,\(\vx_{t}\)的支撑集是整个\(\R^{D}\),所以它有一个我们记为\(p_{t}\)的\textit{密度}(与正文一致)。一个简单的计算表明
\begin{equation}\label{eq:p_t_representation}
    p_{t}(\vxi) = \Ex[\phi_{t}(\vxi - \vx)],
\end{equation}
并且从这个表示中可以推断出(即,从\Cref{prop:diff_convolution}),\(p_{t}\)在\(\vxi\)上是光滑的(即无限可微),在\(t\)上也是光滑的,并且处处为正。这个事实乍一看相当引人注目:即使对于一个完全不规则的随机变量\(\vx\)(例如,一个没有密度的伯努利随机变量),其高斯平滑对于每个(任意小的)\(t > 0\)都存在一个密度。证明留给精通数学分析的读者作为练习。

然而,我们还需要增加一个关于\(\vx\)分布\textit{光滑性}的假设,这将消除在\(t=0\)附近对于低维分布出现的一些技术性问题。\footnote{因为那时各种量会变得高度不规则,处理它们需要大量的额外分析。}尽管如此,我们期望通过额外的工作,我们的结果在更温和的假设下仍然成立。现在,我们假设:
\begin{assumption}\label{assumption:entropy_x_density}
    \(\vx\)有一个二阶连续可微的密度,记为\(p\)。
\end{assumption}


\subsection{扩散过程熵随时间增加}\label{sub:diffusion_entropy_increases}

在本附录小节中,我们提供\Cref{thm:diffusion_entropy_increases}的证明。为方便起见,我们将其重述如下。

\begin{theorem}[扩散增加熵]\label{thm:diffusion_entropy_increases}
    设\(\vx\)为任意满足\Cref{assumption:entropy_x_compact_support,assumption:entropy_x_density}的随机变量,并设\((\vx_{t})_{t \in [0, T]}\)为随机过程\eqref{eq:app_additive_gaussian_noise_model}。那么
    \begin{equation}\label{eq:diffusion_entropy_increases}
        h(\vx_{s}) < h(\vx_{t}), \qquad \forall s, t \colon 0 \leq s < t \leq T.
    \end{equation}
\end{theorem}
\begin{proof}
    在开始之前,让我们先解决一个学究式的问题:\eqref{eq:diffusion_entropy_increases}中的不等式在何时有意义?我们将在\Cref{lem:diffusion_entropy_exists}中证明,在我们的假设下,微分熵是良定义的,永远不为\(+\infty\),并且对于\(t > 0\)是有限的,因此\eqref{eq:diffusion_entropy_increases}中的(严格)不等式是有意义的。

    抛开学究式的讨论,此证明的关键在于表明\(\vx_{t}\)的密度\(p_{t}\)满足一个特定的偏微分方程,该方程与\textit{热传导方程}非常相似。热传导方程是一个著名的偏微分方程,描述了热量在空间中的扩散。这在直觉上是合理的,并描绘了一幅生动的画面:随着时间\(t\)的增加,来自原始(可能高度集中的)\(\vx\)的概率像热量从真空中的热源辐射一样,散布到整个\(\R^{D}\)空间。
    
    这类关于\(p_{t}\)的偏微分方程,对于更一般的随机过程被称为\textit{福克-普朗克方程},是非常强大的工具,因为它们允许我们用\(p_{t}\)的瞬时空间导数来描述其瞬时时间导数,反之亦然,从而为\(p_{t}\)的正则性和动力学提供了简洁的描述。一旦我们得到了\(p_{t}\)的动力学方程,我们就可以利用这个系统得到另一个描述\(h(\vx_{t})\)动力学的方程,毕竟\(h(\vx_{t})\)只是\(p_{t}\)的一个泛函。

    该偏微分方程的描述涉及一个称为拉普拉斯算子\(\Delta\)的数学对象。回想一下你在多元微积分课上学到的,作用于一个时间上可微、空间上二阶可微的函数\(f \colon (0, T) \times \R^{D} \to \R\)的拉普拉斯算子由下式给出
    \begin{equation}
        \Delta f_{t}(\vxi) = \tr(\nabla^{2}f_{t}(\vxi)) = \sum_{i = 1}^{D}\pdv[order=2]{f_{t}}{\xi_{i}}(\vxi).
    \end{equation}
    
    也就是说,通过使用\(p_{t}\)的积分表示并在积分号下求导,我们可以计算出\(p_{t}\)的导数(这在\Cref{prop:p_t_derivatives}中完成),并观察到\(p_{t}\)满足类似热传导的偏微分方程
    \begin{equation}
        \pdv{p_{t}}{t}(\vxi) = t\Delta p_{t}(\vxi).
    \end{equation}
    然后为了找到\(h(\vx_{t})\)的动力学,我们可以再次使用\Cref{prop:dutis}以及类似热传导的偏微分方程来得到
    \begin{align}
        \odv*{h(\vx_{t})}{t}
        &= -\odv*{\int_{\R^{D}}p_{t}(\vxi)\log p_{t}(\vxi)\odif{\vxi}}{t} \\
        &= -\int_{\R^{D}}\pdv*{\bs{p_{t}(\vxi)\log p_{t}(\vxi)}}{t}\odif{\vxi} \\
        &= -\int_{\R^{D}}\pdv{p_{t}}{t}(\vxi)[1 + \log p_{t}(\vxi)]\odif{\vxi} \\
        &= -t\int_{\R^{D}}\Delta p_{t}(\vxi)[1 + \log p_{t}(\vxi)]\odif{\vxi}.
    \end{align}
    通过使用一个稍微复杂的分部积分论证(\Cref{lem:diffusion_ibp}),我们得到
    \begin{align}
        \odv*{h(\vx_{t})}{t}
        &= t\int_{\R^{D}}\ip{\nabla \log p_{t}(\vxi)}{\nabla p_{t}(\vxi)}\odif{\vxi} \\
        &= t\int_{\R^{D}}\frac{\norm{\nabla p_{t}(\vxi)}_{2}^{2}}{p_{t}(\vxi)}\odif{\vxi} \\
        &> 0
    \end{align}
    其中最后一行严格不等式成立,因为若不等式不成立,\(\nabla p_{t}(\vxi)\)需要几乎处处为零(即,除了可能在一个零体积集合上),但这将意味着\(p_{t}\)几乎处处为常数,这与\(p_{t}\)是一个密度的事实相矛盾。

    为了完成证明,我们只需使用微积分基本定理
    \begin{equation}
        h(\vx_{t}) = h(\vx_{s}) + \int_{s}^{t}\odv*{h(\vx_{u})}{u}\odif{u} > h(\vx_{s}),
    \end{equation}
    这就证明了我们的论断。(请注意,当\(h(\vx_{s}) = -\infty\)时,这个式子没有意义,这种情况只可能在\(s=0\)且\(h(\vx)=-\infty\)时发生,但在这种情况下\(h(\vx_{t}) > -\infty\),所以该论断无论如何都是不言自明的。)
\end{proof}

\subsection{去噪过程熵随时间减少}\label{sub:denoising_entropy_decreases}

回想一下,在\Cref{sub:intro_diffusion_denoising}中,我们从随机变量\(\vx_{T}\)开始,并使用如下形式的迭代来对其进行去噪
\begin{equation}\label{eq:app_denoising_iteration}
    \hat{\vx}_{s} := \Ex[\vx_{s} \mid \vx_{t} = \hat{\vx}_{t}] = \frac{s}{t}\hat{\vx}_{t} + \bp{1 - \frac{s}{t}}\bar{\vx}^{\ast}(t, \hat{\vx}_{t}).
\end{equation}
对于\(s, t \in \{t_{0}, t_{1}, \dots, t_{L}\}\)且满足\(s < t\)和\(\vx_{T} = \hat{\vx}_{T}\)。我们希望证明\(h(\hat{\vx}_{s}) < h(\hat{\vx}_{t})\),从而表明去噪过程确实减少了熵。

在进行证明之前,我们对问题陈述做几点说明。首先,Tweedie公式\eqref{eq:tweedie}表明
\begin{equation}
    \bar{\vx}^{\ast}(t, \vx_{t}) = \vx_{t} + t^{2}\nabla p_{t}(\vx_{t}),
\end{equation}
这将从时间\(t\)到时间\(0\)的完整去噪步骤比作在\(\vx_{t}\)的对数密度上进行的一个梯度步。我们能否为\eqref{eq:app_denoising_iteration}中从时间\(t\)到时间\(s\)的完整去噪步骤得到类似的结果?事实证明,我们确实可以,而且非常简单。通过使用\eqref{eq:app_denoising_iteration}和Tweedie公式\eqref{eq:tweedie},我们得到
\begin{equation}\label{eq:app_denoising_iteration_score}
    \Ex[\vx_{s} \mid \vx_{t}] = \frac{s}{t}\vx_{t} + \bp{1 - \frac{s}{t}}\bp{\vx_{t} + t^{2}\nabla_{\vx_{t}}\log p_{t}(\vx_{t})} = \vx_{t} + \bp{1 - \frac{s}{t}}t^{2}\nabla_{\vx_{t}}\log p_{t}(\vx_{t}).
\end{equation}
所以这个迭代去噪步骤又是在扰动后的对数密度\(\log p_{t}\)上的一个梯度步,步长有所缩减。特别地,这一步可以看作是通过\textit{分数函数向量场}对随机变量\(\vx_{t}\)的分布进行扰动,这暗示了与随机微分方程(SDEs)和扩散模型理论\cite{song2020score}的联系。实际上,可以使用这种强大的工具和极限论证来推导以下结果\Cref{thm:conditioning_reduces_entropy}的证明(例如,遵循\cite{DBLP:conf/iclr/ChenC0LSZ23}的阐述中的技术方法)。我们将在这里给出一个更简单的证明,它只使用基本工具,从而阐明通过去噪减少熵过程背后的一些关键量。另一方面,由于\Cref{thm:conditioning_reduces_entropy}中的去噪过程并\textit{不}完全对应于生成观测值\(\vx_{t}\)的加噪过程的\textit{逆}过程,我们将需要处理一些略带技术性的计算。\footnote{对于熟悉扩散模型的读者,我们这里指的是时间反向的前向过程与由\Cref{thm:conditioning_reduces_entropy}定义的过程生成的迭代序列不一致。这些过程在某个极限下是一致的,即当\Cref{thm:conditioning_reduces_entropy}的无限多步以无限小的噪声水平在每一步添加时;对于一般的有限步,无论我们工具的复杂程度如何,我们都必须引入一些近似。}

我们想要证明\(h(\Ex[\vx_{s} \mid \vx_{t}]) < h(\vx_{t})\),即,形式化地:
\begin{theorem}\label{thm:conditioning_reduces_entropy}
    设\(\vx\)为任意满足\Cref{assumption:entropy_x_compact_support,assumption:entropy_x_density}的随机变量,并设\((\vx_{t})_{t \in [0, T]}\)为随机过程\eqref{eq:app_additive_gaussian_noise_model}。那么
    \begin{equation}
        h(\Ex[\vx_{s} \mid \vx_{t}]) < h(\vx_{t}), \qquad \forall s, t \in [0, T] \colon \quad 0 < t \leq \frac{R}{\sqrt{2D}}, \quad 0 \leq s  < t\cdot\min\bc{1, \frac{R^{2}/D - 2t^{2}}{R^{2}/D - t^{2}}}.
    \end{equation}
\end{theorem}
\begin{proof}
    此证明使用两个主要思想:
    \begin{enumerate}
        \item 首先,使用变量替换公式写出\(\Ex[\vx_{s} \mid \vx_{t}]\)的密度。
        \item 其次,对该密度进行界定以控制熵。
    \end{enumerate}
    变量替换的合理性由\Cref{cor:gribonval_A2}保证,该结论最初在\cite{Gribonval2011-pf}中推导得出。

    我们现在来执行这些思想。从\Cref{cor:gribonval_A2}可知,定义为\(\bar{\vx}(\vxi) := \Ex[\vx_{s} \given \vx_{t} = \vxi]\)的函数\(\bar{\vx}\)是可微的、单射的,因此在其值域上是可逆的,我们此后将其值域记为\(\cX \subseteq \R^{D}\)。我们将其逆记为\(\bar{\vx}^{-1}\)。使用变量替换公式,\(\bar{\vx}(\vx_{t})\)的密度\(\bar{p}\)由下式给出
    \begin{equation}
        \bar{p}(\vxi) := \frac{(p_{t} \circ \bar{\vx}^{-1})(\vxi)}{\det(\bar{\vx}^{\prime}(\bar{\vx}^{-1}(\vxi)))},
    \end{equation}
    其中(回想一下,从\Cref{app:optimization}可知)\(\bar{\vx}^{\prime}\)是\(\bar{\vx}\)的雅可比矩阵。由于从\Cref{lem:gribonval_A1}我们知道\(\bar{\vx}^{\prime}\)是一个正定矩阵,其行列式为正,因此整个密度为正。于是有
    \begin{align}
        h(\bar{\vx}(\vx_{t}))
        &= -\int_{\cX}\frac{(p_{t} \circ \bar{\vx}^{-1})(\vxi)}{\det(\bar{\vx}^{\prime}(\bar{\vx}^{-1}(\vxi)))} \log \frac{(p_{t} \circ \bar{\vx}^{-1})(\vxi)}{\det(\bar{\vx}^{\prime}(\bar{\vx}^{-1}(\vxi)))} \odif{\vxi} \\ 
        &= -\int_{\cX}\frac{(p_{t} \circ \bar{\vx}^{-1})(\vxi)}{\det(\bar{\vx}^{\prime}(\bar{\vx}^{-1}(\vxi)))} \log((p_{t} \circ \bar{\vx}^{-1})(\vxi))\odif{\vxi} \\ 
        &\qquad + \int_{\cX}\frac{(p_{t} \circ \bar{\vx}^{-1})(\vxi)}{\det(\bar{\vx}^{\prime}(\bar{\vx}^{-1}(\vxi)))}\logdet\rp{\bar{\vx}^{\prime}(\bar{\vx}^{-1}(\vxi))}\odif{\vxi} \\ 
        &= -\int_{\R^{D}}p_{t}(\vxi)\log p_{t}(\vxi)\odif{\vxi} + \int_{\cX}\frac{(p_{t} \circ \bar{\vx}^{-1})(\vxi)}{\det(\bar{\vx}^{\prime}(\bar{\vx}^{-1}(\vxi)))}\logdet\rp{\bar{\vx}^{\prime}(\bar{\vx}^{-1}(\vxi))}\odif{\vxi} \\ 
        &= h(\vx_{t}) - \int_{\cX}\frac{(p_{t} \circ \bar{\vx}^{-1})(\vxi)}{\det(\bar{\vx}^{\prime}(\bar{\vx}^{-1}(\vxi)))}\log\rp{\frac{1}{\det(\bar{\vx}^{\prime}(\bar{\vx}^{-1}(\vxi)))}}\odif{\vxi}.
    \end{align}
    我们将研究最后一项(包括负号),并证明其为负。

    根据凹性,对于每个\(x \geq 0\),有\(-x\log x \leq 1 - x\)。因此
    \begin{align}
        h(\bar{\vx}(\vx_{t})) - h(\vx_{t})
        &= - \int_{\cX}\frac{(p_{t} \circ \bar{\vx}^{-1})(\vxi)}{\det(\bar{\vx}^{\prime}(\bar{\vx}^{-1}(\vxi)))}\log\rp{\frac{1}{\det(\bar{\vx}^{\prime}(\bar{\vx}^{-1}(\vxi)))}}\odif{\vxi} \\ 
        &\leq  \int_{\cX}(p_{t} \circ \bar{\vx}^{-1})(\vxi)\cdot \bp{1 - \frac{1}{\det(\bar{\vx}^{\prime}(\bar{\vx}^{-1}(\vxi)))}}\odif{\vxi} \\ 
        &= \int_{\cX}(p_{t} \circ \bar{\vx}^{-1})(\vxi)\odif{\vxi} - \int_{\cX}\frac{(p_{t} \circ \bar{\vx}^{-1})(\vxi)}{\det(\bar{\vx}^{\prime}(\bar{\vx}^{-1}(\vxi)))}\odif{\vxi} \\
        &= \int_{\R^{D}}p_{t}(\vxi)\det\rp{\bar{\vx}^{\prime}(\bar{\vx}^{-1}(\vxi))}\odif{\vxi} - \int_{\cX}\bar{p}(\vxi)\odif{\vxi} \\
        &= \int_{\R^{D}}p_{t}(\vxi)\det\rp{\vI + \bp{1 - \frac{s}{t}}t^{2}\nabla^{2}\log p_{t}(\vxi)}\odif{\vxi} - 1.
    \end{align}
    现在,通过对特征值使用算术-几何平均不等式(AM-GM),我们对于任何对称正定矩阵\(\vM \in \PSD(D)\)有如下界
    \begin{equation}
        \det(\vM)^{1/D} = \prod_{i = 1}^{D}\lambda_{i}(\vM)^{1/D} \leq \frac{\sum_{i = 1}^{D}\lambda_{i}(\vM)}{D} = \frac{\tr(\vM)}{D},
    \end{equation}
    我们可以将其应用于上述表达式并得到
    \begin{align}
        &\int_{\R^{D}}p_{t}(\vxi)\det\rp{\vI + \bp{1 - \frac{s}{t}}t^{2}\nabla^{2}\log p_{t}(\vxi)}\odif{\vxi} \\
        &\leq \int_{\R^{D}} p_{t}(\vxi) \tr\rp{\frac{1}{D}\bs{\vI + \bp{1 - \frac{s}{t}}t^{2}\nabla^{2}\log p_{t}(\vxi)}}^{D}\odif{\vxi} \\
        &= \int_{\R^{D}} p_{t}(\vxi)\bp{1 + \frac{\bp{1 - \frac{s}{t}}t^{2}}{D}\tr(\nabla^{2}\log p_{t}(\vxi))}^{D}\odif{\vxi} \\
        &= \int_{\R^{D}} p_{t}(\vxi)\bp{1 + \frac{\bp{1 - \frac{s}{t}}t^{2}}{D}\Delta \log p_{t}(\vxi)}^{D}\odif{\vxi}.
    \end{align}
    从\Cref{lem:app_diffusion_laplacian_control}可知,(其中,回想一下,\(R\)是\(\vx\)支撑集的半径,如\Cref{assumption:entropy_x_compact_support}中所定义)
    \begin{equation}
        \abs{\Delta \log p_{t}(\vxi)} \leq \max\rp{\frac{D}{t^{2}}, \abs*{\frac{R^{2}}{t^{4}} - \frac{D}{t^{2}}}} =: U_{t}.
    \end{equation}
    于是有
    \begin{equation}
        -\frac{\bp{1 - \frac{s}{t}}t^{2}}{D}U_{t} \leq \frac{\bp{1 - \frac{s}{t}}t^{2}}{D}\Delta \log p_{t}(\vxi) \leq \frac{\bp{1 - \frac{s}{t}}t^{2}}{D}U_{t}.
    \end{equation}
    同时,函数\(x \mapsto (1 + x)^{D}\)在\([-1, \infty)\)上是凸的,因此对于
    \(-(1-s/t)t^{2}U_{t}/D \leq x \leq (1-s/t)t^{2}U_{t}/D\)我们有
    \begin{align}
        (1 + x)^{d} 
        &\leq \bp{1 - \frac{\bp{1 - \frac{s}{t}}t^{2}U_{t}}{D}}^{D} + \underbrace{\bs{\bp{1 + \frac{\bp{1 - \frac{s}{t}}t^{2}U_{t}}{D}}^{D} - \bp{1 - \frac{\bp{1 - \frac{s}{t}}t^{2}U_{t}}{D}}^{D}}}_{M(s, t, D)}x \\ 
        &\leq 1 + M(s, t, D)x.
    \end{align}
    这里\(M(s, t, D) > 0\)因为\(U_{t} > 0\)。在上述界中,我们需要验证\(x\)的下界\(\geq -1\)。确实,
    \begin{align}
        -\frac{\bp{1 - \frac{s}{t}}t^{2}}{D}U_{t}
        &= -\frac{\bp{1 - \frac{s}{t}}t^{2}}{D}\max\rp{\frac{D}{t^{2}}, \abs*{\frac{R^{2}}{t^{4}} - \frac{D}{t^{2}}}} \\ 
        &= -\bp{1 - \frac{s}{t}}\max\rp{1, \abs*{\frac{R^{2}}{Dt^{2}} - 1}}
    \end{align}
    注意到,这\(\geq -1\)当且仅当\(\bp{1 - \frac{s}{t}}\cdot\bp{\frac{R^{2}}{Dt^{2}} - 1} \geq 1\),即\(0 < t < R/\sqrt{2D}\)且\(0 \leq s \leq t\cdot\frac{R^{2}/D - 2t^{2}}{R^{2}/D - t^{2}}\),这由假设所保证。
    
    应用这个界,我们得到
    \begin{align}
        &\int_{\R^{D}} p_{t}(\vxi)\bp{1 + \frac{\bp{1 - \frac{s}{t}}t^{2}}{D}\Delta \log p_{t}(\vxi)}^{D}\odif{\vxi} \\ 
        &\leq \int_{\R^{D}}p_{t}(\vxi)\bp{1 + M(s, t, D)\Delta \log p_{t}(\vxi)}\odif{\vxi} \\
        &= 1 + M(s, t, D)\int_{\R^{D}}p_{t}(\vxi)\Delta \log p_{t}(\vxi)\odif{\vxi} \\
        &= 1 - M(s, t, D)\int_{\R^{D}}\ip{\nabla p_{t}(\vxi)}{\nabla\log p_{t}(\vxi)}\odif{\vxi} \\
        &= 1 - M(s, t, D)\int_{\R^{D}}\frac{\norm{\nabla p_{t}(\vxi)}_{2}^{2}}{p_{t}(\vxi)}\odif{\vxi},
    \end{align}
    其中最后几行与\Cref{thm:diffusion_entropy_increases}的证明中相同。将此结果与我们之前的估计相结合,
    \begin{equation}
        h(\bar{\vx}(\vx_{t})) - h(\vx_{t}) \leq - M(s, t, D)\int_{\R^{D}}\frac{\norm{\nabla p_{t}(\vxi)}_{2}^{2}}{p_{t}(\vxi)}\odif{\vxi} < 0
    \end{equation}
    其中不等式是严格的,论证与\Cref{thm:diffusion_entropy_increases}中相同。
\end{proof}

请注意,\(s\)和\(t\)的界依赖于数据分布的半径\(R\),不像\Cref{thm:diffusion_entropy_increases}中的界那样具有普遍性。这个结果实际上在以下意义上是“足够通用”的。注意,如果\(\vx\)有一个二阶连续可微的密度,其支撑集是以\(\vzero\)为中心、半径为\(R\)的球,那么对于\(2R\)、\(3R\)等,即对于任何半径为\(R^{\prime} > R\)的球,它也成立。因此,获得适当去噪保证的一个策略是:固定数据维度\(D\)和离散化方案,然后在分析中将数据半径\(R\)设置得非常大,以使每个去噪步骤都满足\Cref{thm:conditioning_reduces_entropy}中给出的熵减少的要求。这样,去噪过程的每一步确实会如期望的那样减少熵。

 
\subsection{技术性引理与中间结果}\label{sub:app_diffusion_intermediate_results}

在本小节中,我们介绍为我们两个主要概念性定理提供支持的技术性结果。我们的表述将或多或少遵循数学的标准;我们将首先从更高层次的结果开始,然后逐渐回到更具增量性的结果。高层次的结果将使用增量性的结果,这样我们就得到了一个易于阅读的结果依赖顺序:没有结果依赖于它之前的结果。互不依赖的结果通常按照它们在上述两个证明中出现的顺序排列。


\subsubsection{微分熵的有限性}

我们首先证明熵在随机过程中存在且是有限的。

\begin{lemma}\label{lem:diffusion_entropy_exists}
    设\(\vx\)为任意随机变量,并设\((\vx_{t})_{t \in [0, T]}\)为随机过程\eqref{eq:app_additive_gaussian_noise_model}。
    \begin{enumerate}
        \item 对于\(t > 0\),微分熵\(h(\vx_{t})\)存在且\(> -\infty\)。
        \item 如果\(\vx\)还满足\Cref{assumption:entropy_x_compact_support},那么\(h(\vx) < \infty\)且\(h(\vx_{t})\ < \infty\)。
    \end{enumerate}
\end{lemma}
\begin{proof}
    为证明\Cref{lem:diffusion_entropy_exists}.1,我们使用一个经典但繁琐的分析论证。由于\(\vx_{t}\)有密度,我们可以写出
    \begin{equation}
        h(\vx_{t}) = -\int_{\R^{D}}p_{t}(\vxi)\log p_{t}(\vxi) \odif{\vxi}.
    \end{equation}
    相应地,令\(g \colon \R^{D} \to \R\)定义为
    \begin{equation}
        g(\vxi) := -p_{t}(\vxi)\log p_{t}(\vxi) \implies h(\vx_{t}) = \int_{\R^{D}}g(\vxi)\odif{\vxi}.
    \end{equation}
    按照分析学中界定积分值的常规方法,定义
    \begin{equation}
        g_{+}(\vxi) := \max(g(\vxi), 0), \quad g_{-}(\vxi) := \max(-g(\vxi), 0) \quad \implies \quad g = g_{+} - g_{-}\quad \text{and} \quad g_{+}, g_{-} \geq 0.
    \end{equation}
    那么
    \begin{equation}
        h(\vx_{t}) = \int_{\R^{D}}g_{+}(\vxi)\odif{\vxi} - \int_{\R^{D}}g_{-}(\vxi)\odif{\vxi},
    \end{equation}
    并且由于被积函数非负,这两个积分都保证是非负的。
    
    为了证明\(h(\vx_{t})\)是良定义的,我们需要证明\(\int_{\R^{D}}g_{+}(\vxi)\odif{\vxi} < \infty\)或\(\int_{\R^{D}}g_{-}(\vxi) \odif{\vxi} < \infty\)。为了证明\(h(\vx_{t}) > -\infty\),只需证明\(\int_{\R^{D}}g_{-}(\vxi)\odif{\vxi} < \infty\)即可。为了界定\(g_{-}\)的积分,我们需要理解\(g_{-}\)这个量,即,我们想刻画\(g\)何时为负。
    \begin{equation}
        g(\vxi) \leq 0 \iff p_{t}(\vxi)\log p_{t}(\vxi) \geq 0 \iff \log p_{t}(\vxi) \geq 0 \iff p_{t}(\vxi) \geq 1.
    \end{equation}
    因此,我们有
    \begin{equation}
        g_{-}(\vxi) = \indvar(p_{t}(\vxi) \geq 1)\cdot (-g(\vxi)) = \indvar(p_{t}(\vxi) \geq 1)p_{t}(\vxi)\log p_{t}(\vxi).
    \end{equation}
    为了界定\(g_{-}(\vxi)\)的积分,我们需要证明\(p_{t}\)“不太集中”,即\(p_{t}\)不太大。为了证明这一点,在这种情况下,我们很幸运能够直接界定函数\(g_{-}(\vxi)\)本身。也就是说,注意到
    \begin{equation}
        \max_{\vxi \in \R^{D}}\phi_{t}(\vxi - \vx) = \phi_{t}(\vzero) = \frac{1}{(2\pi)^{D/2}t^{D}} =: C_{t}.
    \end{equation}
    当\(t \to 0\)时它会趋于无穷,但对于所有有限的\(t\)都是有限的。因此
    \begin{equation}
        p_{t}(\vxi) = \Ex \phi_{t}(\vxi - \vx) \leq \Ex C_{t} = C_{t}.
    \end{equation}
    现在有两种情况。
    \begin{itemize}
        \item 如果\(C_{t} < 1\),那么\(p_{t}(\vxi) < 1\),所以指示函数永远不为\(1\),因此\(g_{-}\)恒等于\(0\),其积分也为\(0\)。
        \item 如果\(C_{t} \geq 1\),那么\(\log C_{t} \geq 0\),所以由于对数函数是单调递增的,
        \begin{align}
            \int_{\R^{D}}g_{-}(\vxi)\odif{\vxi}
            &= \int_{\R^{D}}\indvar(p_{t}(\vxi) \geq 1)p_{t}(\vxi)\log p_{t}(\vxi)\odif{\vxi}  \\ 
            &= \Ex[\indvar(p_{t}(\vx_{t}) \geq 1)\log p_{t}(\vx_{t})]  \\ 
            &\leq \Ex[\indvar(p_{t}(\vx_{t}) \geq 1) \log C_{t}] \\ 
            &= \Pr[p_{t}(\vx_{t}) \geq 1]\log C_{t}.
        \end{align}
    \end{itemize}
    因此我们有\(\int_{\R^{D}}g_{-}(\vxi)\odif{\vxi} < \infty\),所以微分熵\(h(\vx_{t})\)存在且\(> -\infty\)。

    为证明\Cref{lem:diffusion_entropy_exists}.2,假设\Cref{assumption:entropy_x_compact_support}成立。我们想证明\(h(\vx) < \infty\)且\(h(\vx_{t}) < \infty\)。实现这一点的机制是相同的,并且涉及最大熵结果\Cref{thmx:max_entropy}。也就是说,由于\(\vx\)是绝对有界的,它有一个有限的协方差,我们记为\(\vSigma\)。那么\(\vx_{t}\)的协方差是\(\vSigma + t^{2}\vI\)。因此,\(\vx\)和\(\vx_{t}\)的熵可以由具有相应协方差的正态分布的熵来上界,即\(\log[(2\pi e)^{D}\det(\vSigma)]\)或\(\log[(2\pi e)^{D}\det(\vSigma + t^{2}\vI)]\),两者都\(< \infty\)。
\end{proof}

\subsubsection{德布鲁因恒等式中的分部积分}

最后,我们补充在\Cref{thm:diffusion_entropy_increases,thm:conditioning_reduces_entropy}的证明中提到的分部积分论证。这个论证在概念上相当简单,但需要一些技术性的估计来证明分部积分中的边界项消失。

\begin{lemma}\label{lem:diffusion_ibp}
    设\(\vx\)为任意满足\Cref{assumption:entropy_x_compact_support,assumption:entropy_x_density}的随机变量,并设\((\vx_{t})_{t \in [0, T]}\)为随机过程\eqref{eq:app_additive_gaussian_noise_model}。对于\(t \geq 0\),设\(p_{t}\)为\(\vx_{t}\)的密度。那么对于一个常数\(c \in \R\),有
    \begin{equation}
        \int_{\R^{D}}\Delta p_{t}(\vxi)[c + \log p_{t}(\vxi)]\odif{\vxi} = -\int_{\R^{D}}\ip{\nabla \log p_{t}(\vxi)}{\nabla p_{t}(\vxi)}\odif{\vxi}.
    \end{equation}
\end{lemma}
\begin{proof}
    此证明的基本思想分为两步:
    \begin{itemize}
        \item 首先,应用格林定理在紧集上进行分部积分;
        \item 其次,将这个紧集的半径趋于\(+\infty\),从而得到在整个\(\R^{D}\)上的积分。
    \end{itemize}
    格林定理表明,对于任何紧集\(\cK \subseteq \R^{D}\)、二阶连续可微的\(\plainphi \colon \R^{D} \to \R\)和连续可微的\(\psi \colon \R^{D} \to \R\),
    \begin{equation}
        \int_{\cK}\bc{\psi(\vxi) \Delta \plainphi(\vxi) + \ip{\nabla \psi(\vxi)}{\nabla \plainphi(\vxi)}}\odif{\vxi} = \int_{\partial \cK}\psi(\vxi)\ip{\nabla \plainphi(\vxi)}{\vn(\vxi)}\odif{\sigma(\vxi)}
    \end{equation}
    其中\(\odif{\sigma(\vxi)}\)表示对“表面测度”的积分,即在\(\cK\)的边界\(\partial \cK\)上的诱导测度,相应地\(\vxi\)在该表面上取值,\(\vn(\vxi)\)是\(\cK\)在表面点\(\vxi\)处的单位法向量。现在,取\(\plainphi(\vxi) := p_{t}(\vxi)\)和\(\psi(\vxi) := c + \log p_{t}(\vxi)\),在一个以\(\vzero\)为中心、半径为\(r > 0\)的球\(B_{r}(\vzero)\)上(因此\(\partial B_{r}(\vzero)\)是以\(\vzero\)为中心、半径为\(r\)的球面,且\(\vn(\vxi) = \vxi/\norm{\vxi}_{2} = \vxi/r\)):
    \begin{align}
        &\int_{B_{r}(\vzero)}\bc{\Delta p_{t}(\vxi)[c + \log p_{t}(\vxi)] + \ip{\nabla \log p_{t}(\vxi)}{\nabla p_{t}(\vxi)}}\odif{\vxi} \\
        &= \int_{\partial B_{r}(\vzero)}[c + \log p_{t}(\vxi)]\ip*{\nabla p_{t}(\vxi)}{\frac{\vxi}{r}}\odif{\sigma(\vxi)} \\
        &= \frac{1}{r}\int_{\partial B_{r}(\vzero)}[c + \log p_{t}(\vxi)]\ip*{\nabla p_{t}(\vxi)}{\vxi}\odif{\sigma(\vxi)}.
    \end{align}
    令\(r \to \infty\),我们有
    \begin{align}
        &\int_{\R^{D}}\bc{\Delta p_{t}(\vxi)[c + \log p_{t}(\vxi)] + \ip{\nabla \log p_{t}(\vxi)}{\nabla p_{t}(\vxi)}}\odif{\vxi} \label{eq:app_diffusion_r_infinity_limit} \\
        &= \lim_{r \to \infty}\int_{B_{r}(\vzero)}\bc{\Delta p_{t}(\vxi)[c + \log p_{t}(\vxi)] + \ip{\nabla \log p_{t}(\vxi)}{\nabla p_{t}(\vxi)}}\odif{\vxi} \\
        &= \lim_{r \to \infty}\int_{\partial B_{r}(\vzero)}\bc{\Delta p_{t}(\vxi)[c + \log p_{t}(\vxi)] + \ip{\nabla \log p_{t}(\vxi)}{\nabla p_{t}(\vxi)}}\odif{\vxi} \\ 
        &= \lim_{r \to \infty}\frac{1}{r}\int_{\partial B_{r}(\vzero)}[c + \log p_{t}(\vxi)]\ip*{\nabla p_{t}(\vxi)}{\vxi}\odif{\sigma(\vxi)},
    \end{align}
    其中第一个不等式由被积函数的控制收敛定理得出。剩下的任务是计算最后一个极限。为此,我们对每一项进行渐近展开。主要方法如下:对于\(\vxi \in \partial B_{r}(\vzero)\),我们有\(\norm{\vxi}_{2} = r\),所以
    \begin{align}
        p_{t}(\vxi)
        &= \Ex[\phi_{t}(\vxi - \vx)] \\ 
        &= \Ex\rs{\underbrace{\frac{1}{(2\pi)^{D/2}t^{D}}}_{:= C_{t}}e^{-\|\vxi - \vx\|_{2}^{2}/(2t^{2})}} \\
        &= C_{t}\Ex\rs{e^{-(\norm{\vxi}_{2}^{2} - 2\ip{\vxi}{\vx} + \norm{\vx}_{2}^{2})/(2t^{2})}} \\ 
        &= C_{t}\Ex\rs{e^{-(r^{2} - 2\ip{\vxi}{\vx} + \norm{\vx}_{2}^{2})/(2t^{2})}} \\ 
        &= C_{t}e^{-r^{2}/(2t^{2})}\Ex[e^{(2\ip{\vxi}{\vx} - \norm{\vx}_{2}^{2})/(2t^{2})}]. 
    \end{align}
    注意,因为\(\norm{\vxi}_{2} = r\),由柯西-施瓦茨不等式我们有
    \begin{equation}
        -2r\norm{\vx}_{2} - \norm{\vx}_{2}^{2} \leq 2\ip{\vxi}{\vx} - \norm{\vx}_{2}^{2} \leq 2r\norm{\vx}_{2} - \norm{\vx}_{2}^{2}.
    \end{equation}
    回想一下,根据\Cref{assumption:entropy_x_compact_support},\(\vx\)的支撑集是半径为\(R\)的紧集\(\cS\)。因此
    \begin{equation}
        -2R(r + R) \leq 2\ip{\vxi}{\vx} - \norm{\vx}_{2}^{2} \leq 2Rr.
    \end{equation}
    换句话说,我们有
    \begin{equation}
        C_{t}e^{-[r^{2} + 2R(r + R)]/(2t^{2})} \leq p_{t}(\vxi) \leq C_{t}e^{[-r^{2} + 2Rr]/(2t^{2})}.
    \end{equation}
    现在为了计算梯度,我们可以使用\Cref{prop:p_t_derivatives}和期望的线性性质来计算
    \begin{align}
        \ip{\nabla p_{t}(\vxi)}{\vxi}
        &= \ip*{-\frac{1}{t^{2}}\Ex\rs{\bp{\vxi - \vx}\phi_{t}(\vxi - \vx)}}{\vxi} \\
        &= -\frac{1}{t^{2}}\Ex\rs{\ip{\vxi - \vx}{\vxi}\phi_{t}(\vxi - \vx)} \\
        &= -\frac{1}{t^{2}}\Ex\rs{\bp{\norm{\vxi}_{2}^{2} - \ip{\vxi}{\vx}}\phi_{t}(\vxi - \vx)} \\
        &= -\frac{1}{t^{2}}\Ex\rs{\bp{r^{2} - \ip{\vxi}{\vx}}\phi_{t}(\vxi - \vx)} \\
        &= \frac{1}{t^{2}}\Ex\rs{\bp{\ip{\vxi}{\vx} - r^{2}}\phi_{t}(\vxi - \vx)}.
    \end{align}
    再次使用柯西-施瓦茨不等式和表示\(p_{t}(\vxi) := \Ex[\phi_{t}(\vxi - \vx)]\),我们有
    \begin{align}
        &\frac{1}{t^{2}}\Ex\rs{\bp{-Rr - r^{2}}\phi_{t}(\vxi - \vx)} \leq \ip{\nabla p_{t}(\vxi)}{\vxi} \leq \frac{1}{t^{2}}\Ex\rs{\bp{Rr - r^{2}}\phi_{t}(\vxi - \vx)} \\
        &\frac{1}{t^{2}}\bp{-Rr - r^{2}}\Ex\rs{\phi_{t}(\vxi - \vx)} \leq \ip{\nabla p_{t}(\vxi)}{\vxi} \leq \frac{1}{t^{2}}\bp{Rr - r^{2}}\Ex\rs{\phi_{t}(\vxi - \vx)} \\
        &-\frac{r(R + r)}{t^{2}}p_{t}(\vxi) \leq \ip{\nabla p_{t}(\vxi)}{\vxi} \leq -\frac{r(r - R)}{t^{2}}p_{t}(\vxi).
    \end{align}
    对于\(r > R > 0\)(这是合适的,因为我们将取极限\(r \to \infty\)而\(R\)是固定的),两边都是负的。这是合理的:大部分概率质量包含在半径为\(R\)的球内,因此分数函数指向内部,与向外的向量\(\vxi\)的内积为负。因此,使用\(p_{t}(\vxi)\)的适当界,
    \begin{equation}
        -\frac{r(R + r)}{t^{2}}\cdot C_{t}e^{[-r^{2} + 2Rr]/(2t^{2})} \leq \ip{\nabla p_{t}(\vxi)}{\vxi} \leq -\frac{r(r - R)}{t^{2}}\cdot C_{t}e^{-[r^{2} + 2R(r + R)]/(2t^{2})}.
    \end{equation}
    然后,注意到\(C_{t} = \mathrm{poly}(t^{-1})\),我们可以计算
    \begin{equation}
        [c + \log p_{t}(\vxi)]\ip{\nabla p_{t}(\vxi)}{\vxi} = \mathrm{poly}(r, R, t^{-1}, c)e^{-\Theta_{r}(r^{2})}
    \end{equation}
    所以可以看到,设\(\partial B_{r}(\vzero)\)的表面积为\(\omega_{D} r^{D - 1}\),其中\(\omega_{D}\)是\(D\)的函数,我们有
    \begin{equation}
        \frac{1}{r}\int_{\partial B_{r}(\vzero)}[c + \log p_{t}(\vxi)]\ip{\nabla p_{t}(\vxi)}{\vxi}\odif{\vxi} = \mathrm{poly}(r, R, t^{-1}, c)e^{-\Theta_{r}(r^{2})}
    \end{equation}
    因此指数衰减的尾部意味着
    \begin{equation}
        \lim_{r \to \infty}\frac{1}{r}\int_{\partial B_{r}(\vzero)}[c + \log p_{t}(\vxi)]\ip{\nabla p_{t}(\vxi)}{\vxi}\odif{\vxi} = 0.
    \end{equation}
    最后,代入\eqref{eq:app_diffusion_r_infinity_limit},我们有
    \begin{align}
        &\int_{\R^{D}}\bc{\Delta p_{t}(\vxi)[c + \log p_{t}(\vxi)] + \ip{\nabla \log p_{t}(\vxi)}{\nabla p_{t}(\vxi)}}\odif{\vxi} = 0 \\
        \implies 
        &\int_{\R^{D}}\Delta p_{t}(\vxi)[c + \log p_{t}(\vxi)]\odif{\vxi} = -\int_{\R^{D}}\ip{\nabla \log p_{t}(\vxi)}{\nabla p_{t}(\vxi)}\odif{\vxi}
    \end{align}
    如所声称。
\end{proof}

\subsubsection{去噪器\(\bar{\vx}\)的局部可逆性}

这里我们提供在\Cref{thm:conditioning_reduces_entropy}证明中使用的一些结果,它们是\cite{Gribonval2011-pf}中相应结果的适当推广。

\begin{lemma}[\cite{Gribonval2011-pf}引理A.1的推广]\label{lem:gribonval_A1}
    设\(\vx\)为任意满足\Cref{assumption:entropy_x_compact_support,assumption:entropy_x_density}的随机变量,并设\((\vx_{t})_{t \in [0, T]}\)为随机过程\eqref{eq:app_additive_gaussian_noise_model}。设\(s, t \in [0, T]\)满足\(0 \leq s < t \leq T\),并设\(\bar{\vx}(\vxi) := \Ex[\vx_{s} \mid \vx_{t} = \vxi]\)。雅可比矩阵\(\bar{\vx}^{\prime}(\vxi)\)是对称正定的。
\end{lemma}
\begin{proof}
    我们有
    \begin{equation}
        \bar{\vx}^{\prime}(\vxi) = \vI + \bp{1 - \frac{s}{t}}t^{2}\nabla^{2}\log p_{t}(\vxi).
    \end{equation}
    这里我们展开
    \begin{equation}
        \nabla^{2}\log p_{t}(\vxi) = \frac{p_{t}(\vxi)\nabla^{2}p_{t}(\vxi) - (\nabla p_{t}(\vxi))(\nabla p_{t}(\vxi))^{\top}}{p_{t}(\vxi)^{2}}.
    \end{equation}
    所以我们需要确保
    \begin{align}
        \bar{\vx}^{\prime}(\vxi)
        &= \vI + \bp{1 - \frac{s}{t}}t^{2}\frac{p_{t}(\vxi)\nabla^{2}p_{t}(\vxi) - (\nabla p_{t}(\vxi))(\nabla p_{t}(\vxi))^{\top}}{p_{t}(\vxi)^{2}} \\
        &= \frac{p_{t}(\vxi)^{2}\vI + \bp{1 - \frac{s}{t}}t^{2}\bs{p_{t}(\vxi)\nabla^{2}p_{t}(\vxi) - (\nabla p_{t}(\vxi))(\nabla p_{t}(\vxi))^{\top}}}{p_{t}(\vxi)^{2}}
    \end{align}
    是对称半正定的。确实,它显然是对称的(根据克莱罗定理)。为了证明其半正定性,我们代入由\eqref{eq:p_t_representation}给出的\(p_{t}\)的期望表示(以及由\Cref{prop:p_t_derivatives}给出的\(\nabla p_{t}\)、\(\Delta p_{t}\))来得到(其中\(\vx\)如定义,\(\vy\)与\(\vx\)独立同分布),
    \begin{align}
        &\vv^{\top}[\bar{\vx}^{\prime}(\vxi)]\vv \\
        &= p_{t}(\vxi)^{-2}\vv^{\top}\Bigg\{p_{t}(\vxi)^{2}\vI + \bp{1 - \frac{s}{t}}t^{2}\Ex[\phi_{t}(\vxi - \vx)]\Ex\rs{\phi_{t}(\vxi - \vx)\cdot\frac{(\vxi - \vx)(\vxi - \vx)^{\top} - t^{2}\vI}{t^{4}}} \\
        & \qquad \qquad \qquad -\bp{1 - \frac{s}{t}}t^{2}\Ex\rs{-\phi_{t}(\vxi - \vx)\cdot\frac{\vxi - \vx}{t^{2}}}\Ex\rs{-\phi_{t}(\vxi - \vx)\cdot\frac{\vxi - \vx}{t^{2}}}^{\top}\Bigg\}\vv \\
        &= p_{t}(\vxi)^{-2}\vv^{\top}\Bigg\{\Ex[\phi_{t}(\vxi - \vx)\phi_{t}(\vxi - \vy)\vI] \\
        & \qquad \qquad \qquad + \bp{1 - \frac{s}{t}}t^{2}\Ex\rs{\phi_{t}(\vxi - \vx)\phi_{t}(\vxi - \vy)\cdot\frac{(\vxi - \vy)(\vxi - \vy)^{\top} - t^{2}\vI}{t^{4}}} \\
        & \qquad \qquad \qquad -\bp{1 - \frac{s}{t}}t^{2}\Ex\rs{\phi_{t}(\vxi - \vx)\phi_{t}(\vxi - \vy)\cdot\frac{(\vxi - \vx)(\vxi - \vy)^{\top}}{t^{4}}}\Bigg\} \\ 
        &= \frac{1 - s/t}{p_{t}(\vxi)^{2}}\vv^{\top}\Ex\mathopen{}\Bigg[\phi_{t}(\vxi - \vx)\phi_{t}(\vxi - \vy)\bc{\frac{1}{1 - s/t}\vI + \frac{(\vxi - \vy)(\vxi - \vy)^{\top}}{t^{2}} - \vI - \frac{(\vxi - \vx)(\vxi - \vy)^{\top}}{t^{2}}}\Bigg]\vv \\
        &= \frac{t - s}{tp_{t}(\vxi)^{2}}\vv^{\top}\Ex\rs{\frac{s}{t - s}\vI +\frac{(\vxi - \vx)(\vxi - \vx)^{\top}}{2t^{2}} + \frac{(\vxi - \vy)(\vxi - \vy)^{\top}}{2t^{2}} - \frac{(\vxi - \vx)(\vxi - \vy)^{\top}}{t^{2}}}\vv \\
        &= \frac{t - s}{tp_{t}(\vxi)^{2}}\vv^{\top}\Ex\rs{\frac{s}{t - s}\vI +\frac{1}{2t^{2}}\bp{(\vxi - \vx)(\vxi - \vx)^{\top} + (\vxi - \vy)(\vxi - \vy)^{\top} - 2(\vxi - \vx)(\vxi - \vy)^{\top}}}\vv \\
        &= \frac{t - s}{tp_{t}(\vxi)^{2}}\Ex\rs{\frac{s}{t - s}\norm{\vv}_{2}^{2} +\frac{1}{2t^{2}}\bp{[(\vxi - \vx)^{\top}\vv]^{2} + [(\vxi - \vy)^{\top}\vv]^{2} - 2[(\vxi - \vx)^{\top}\vv][(\vxi - \vy)^{\top}\vv]}} \\
        &= \frac{t - s}{tp_{t}(\vxi)^{2}}\Ex\rs{\frac{s}{t - s}\norm{\vv}_{2}^{2} +\frac{1}{2t^{2}}\bp{[(\vxi - \vx)^{\top}\vv]^{2} + [(\vxi - \vy)^{\top}\vv]^{2} - 2[(\vxi - \vx)^{\top}\vv][(\vxi - \vy)^{\top}\vv]}} \\
        &= \frac{t - s}{tp_{t}(\vxi)^{2}}\Ex\rs{\frac{s}{t - s}\norm{\vv}_{2}^{2} +\frac{1}{2t^{2}}\bp{[(\vxi - \vx)^{\top}\vv] - [(\vxi - \vy)^{\top}\vv]}^{2}} \\
        &= \frac{t - s}{tp_{t}(\vxi)^{2}}\Ex\rs{\frac{s}{t - s}\norm{\vv}_{2}^{2} +\frac{1}{2t^{2}}[(\vy - \vx)^{\top}\vv]^{2}} \\
        &= \frac{s}{tp_{t}(\vxi)^{2}}\norm{\vv}_{2}^{2} + \frac{t - s}{2t^{3}p_{t}(\vxi)}\Ex[\{(\vy - \vx)^{\top}\vv\}^{2}]
    \end{align}
    由于\(\vx\)和\(\vy\)是独立同分布的,整个积分(即原始的二次型)为\(0\)当且仅当\(s = 0\)且\(\vx\)的支撑集完全包含在一个与\(\vv\)正交的仿射子空间中。但这被假设排除了(即\(\vx\)在\(\R^{D}\)上有密度),所以雅可比矩阵\(\bar{\vx}^{\prime}(\vxi)\)是对称正定的。
\end{proof}

\begin{lemma}[\cite{Gribonval2011-pf}推论A.2第1部分的推广]\label{lem:gribonval_A2}
    设\(f \colon \R^{D} \to \R^{D}\)为任意可微函数,其雅可比矩阵\(f^{\prime}(\vx)\)是对称正定的。那么\(f\)是单射的,因此作为函数\(\R^{D} \to \Range(f)\)是可逆的,其中\(\Range(f)\)是\(f\)的值域。
\end{lemma}
\begin{proof}
    假设\(f\)不是单射的,即存在\(\vx, \vx^{\prime}\)使得\(f(\vx) = f(\vx^{\prime})\)而\(\vx \neq \vx^{\prime}\)。定义\(\vv := (\vx^{\prime} - \vx)/\norm{\vx^{\prime} - \vx}_{2}\)。定义函数\(g \colon \R \to \R\)为\(g(t) := \vv^{\top}f(\vx + t\vv)\)。那么\(g(0) = \vv^{\top}f(\vx) = \vv^{\top}f(\vx^{\prime}) = g(\norm{\vx^{\prime} - \vx}_{2})\)。由于\(f\)是可微的,\(g\)也是可微的,所以根据中值定理,导数\(g^{\prime}\)必须在某个\(t^{\ast} \in (0, \norm{\vx^{\prime} - \vx}_{2})\)处为零。然而,
    \begin{equation}
        g^{\prime}(t^{\ast}) := \vv^{\top}\bs{f^{\prime}(\vx + t^{\ast}\vv)}\vv > 0
    \end{equation}
    因为雅可比矩阵是正定的。因此我们得到了一个矛盾,如所声称。
\end{proof}

结合以上两个结果,我们得到以下关键结果。

\begin{corollary}[\cite{Gribonval2011-pf}推论A.2第2部分的推广]\label{cor:gribonval_A2}
    设\(\vx\)为任意满足\Cref{assumption:entropy_x_compact_support,assumption:entropy_x_density}的随机变量,并设\((\vx_{t})_{t \in [0, T]}\)为随机过程\eqref{eq:app_additive_gaussian_noise_model}。设\(s, t \in [0, T]\)满足\(0 \leq s < t \leq T\),并设\(\bar{\vx}(\vxi) := \Ex[\vx_{s} \mid \vx_{t} = \vxi]\)。那么\(\bar{\vx}\)是单射的,因此在其值域上是可逆的。
\end{corollary}
\begin{proof}
    唯一剩下要证明的是\(\bar{\vx}\)是可微的,但这从Tweedie公式(\Cref{thm:tweedie})可以立即得出,该公式表明\(\bar{\vx}\)是可微的当且仅当\(\nabla \log p_{t}\)是可微的,而这由\Cref{eq:p_t_representation}提供。
\end{proof}

\subsubsection{控制拉普拉斯算子\(\Delta \log p_{t}\)}

最后,我们推导一个技术性估计,这是\Cref{thm:conditioning_reduces_entropy}证明所必需的,并且实际上激发了对可行\(t\)的假设。

\begin{lemma}\label{lem:app_diffusion_laplacian_control}
    设\(\vx\)为任意满足\Cref{assumption:entropy_x_compact_support,assumption:entropy_x_density}的随机变量,并设\((\vx_{t})_{t \in [0, T]}\)为随机过程\eqref{eq:app_additive_gaussian_noise_model}。设\(p_{t}\)为\(\vx_{t}\)的密度。那么,对于\(t > 0\)有
    \begin{equation}
        \sup_{\vxi \in \R^{D}}\abs{\Delta \log p_{t}(\vxi)} \leq \max\rp{\frac{D}{t^{2}}, \abs*{\frac{R}{t^{4}} - \frac{D}{t^{2}}}}.
    \end{equation}
\end{lemma}
\begin{proof}
    通过链式法则,一个简单的练习可以计算出
    \begin{equation}
        \Delta \log p_{t}(\vxi) = \frac{\Delta p_{t}(\vxi)}{p_{t}(\vxi)} - \frac{\norm{\nabla p_{t}(\vxi)}_{2}^{2}}{p_{t}(\vxi)^{2}}.
    \end{equation}
    使用\Cref{prop:p_t_derivatives}来写出\(\Delta p_{t}(\vxi)\)中的项,我们得到
    \begin{align}
        \frac{\Delta p_{t}(\vxi)}{p_{t}(\vxi)}
        &= \frac{\Ex\rs{\frac{\norm{\vxi - \vx}_{2}^{2} - Dt^{2}}{t^{4}} \cdot \phi_{t}(\vxi - \vx)}}{\Ex[\phi_{t}(\vxi - \vx)]} \\
        &= \frac{\int_{\R^{D}}\bc{\frac{\norm{\vxi - \vu}_{2}^{2} - Dt^{2}}{t^{4}}}\phi_{t}(\vxi - \vu)p(\vu)\odif{\vu}}{\int_{\R^{D}}\phi_{t}(\vxi - \vu)p(\vu)\odif{\vu}}.
    \end{align}
    这看起来像一个贝叶斯边缘化,所以我们定义适当的归一化密度
    \begin{equation}
        q_{\vxi}(\vu) = \frac{\phi_{t}(\vxi - \vu)p(\vu)}{\int_{\R^{D}}\phi_{t}(\vxi - \vv)p(\vv)\odif{\vv}} = \frac{\phi_{t}(\vxi - \vu)p(\vu)}{\Ex[\phi_{t}(\vxi - \vx)]} = \frac{\phi_{t}(\vxi - \vu)p(\vu)}{p_{t}(\vxi)}
    \end{equation}
    然后,定义\(\vy_{\vxi} \sim q_{\vxi}\),我们可以写出
    \begin{equation}
        \frac{\Delta p_{t}(\vxi)}{p_{t}(\vxi)} = \int_{\R^{D}}\bc{\frac{\norm{\vxi - \vu}_{2}^{2} - Dt^{2}}{t^{4}}}q_{\vxi}(\vu)\odif{\vu} = \frac{1}{t^{4}}\Ex[\norm{\vxi - \vy_{\vxi}}_{2}^{2}] - \frac{D}{t^{2}}.
    \end{equation}
    类似地,写出第二项(非平方)我们得到
    \begin{equation}
        \frac{\nabla p_{t}(\vxi)}{p_{t}(\vxi)} = -\frac{\vxi - \Ex[\vy_{\vxi}]}{t^{2}}.
    \end{equation}
    令\(\vz_{\vxi} := \vy_{\vxi} - \vxi\),则有
    \begin{equation}
        \frac{\Delta p_{t}(\vxi)}{p_{t}(\vxi)} = \frac{\Ex[\norm{\vz_{\vxi}}_{2}^{2}]}{t^{4}} - \frac{D}{t^{2}}, \qquad \frac{\nabla p_{t}(\vxi)}{p_{t}(\vxi)} = \frac{\Ex[\vz_{\vxi}]}{t^{2}}.
    \end{equation}
    因此,完全写出\(\Delta \log p_{t}\),我们有
    \begin{align}
        \Delta \log p_{t}(\vxi)
        &= \frac{\Ex[\norm{\vz_{\vxi}}_{2}^{2}]}{t^{4}} - \frac{D}{t^{2}} - \frac{\norm{\Ex[\vz_{\vxi}]}_{2}^{2}}{t^{4}} \\
        &= \frac{\Ex[\norm{\vz_{\vxi}}_{2}^{2}] - \norm{\Ex[\vz_{\vxi}]}_{2}^{2}}{t^{4}} - \frac{D}{t^{2}} \\
        &= \frac{\tr(\Cov(\vz_{\vxi}))}{t^{4}} - \frac{D}{t^{2}} \\
        &= \frac{\tr(\Cov(\vy_{\vxi}))}{t^{4}} - \frac{D}{t^{2}}.
    \end{align}
    这个迹的一个平凡下界是\(0\),因为协方差矩阵是半正定的。为了找到一个上界,注意\(\vy_{\vxi}\)只在\(\vx\)的支撑集内取值(因为\(p\)是\(\vy_{\vxi}\)密度\(q_{\vxi}\)的一个因子),根据\Cref{assumption:entropy_x_compact_support},这是一个半径为\(R := \sup_{\vxi \in \R^{D}}\norm{\vxi}_{2}\)的紧集\(\cS\)。所以
    \begin{equation}
        \tr(\Cov(\vy_{\vxi})) = \Ex[\norm{\vy_{\vxi}}_{2}^{2}] - \norm{\Ex[\vy_{\vxi}]}_{2}^{2} \leq \Ex[\norm{\vy_{\vxi}}_{2}^{2}] \leq R^{2}.
    \end{equation}
    因此
    \begin{equation}
        -\frac{D}{t^{2}} \leq \Delta \log p_{t}(\vxi) \leq \frac{R^{2}}{t^{4}} - \frac{D}{t^{2}},
    \end{equation}
    这就证明了该论断。
\end{proof}

\subsubsection{导数计算}

在这里,我们计算一些有用的导数,它们将在整个附录中被重复使用。

\begin{proposition}\label{prop:p_t_derivatives}
    设\(\vx\)为任意满足\Cref{assumption:entropy_x_compact_support,assumption:entropy_x_density}的随机变量,并设\((\vx_{t})_{t \in [0, T]}\)为随机过程\eqref{eq:app_additive_gaussian_noise_model}。对于\(t \geq 0\),设\(p_{t}\)为\(\vx_{t}\)的密度。那么
    \begin{align}
        \pdv{p_{t}}{t}(\vxi)
        &= \Ex\rs{\phi_{t}(\vxi - \vx)\cdot\frac{\norm{\vxi - \vx}_{2}^{2} - Dt^{2}}{t^{3}}} \\
        \nabla p_{t}(\vxi)
        &= -\Ex\rs{\phi_{t}(\vxi - \vx)\cdot\frac{\vxi - \vx}{t^{2}}} \\
        \nabla^{2}p_{t}(\vxi)
        &= \Ex\rs{\phi_{t}(\vxi - \vx)\cdot\frac{(\vxi - \vx)(\vxi - \vx)^{\top} - t^{2}\vI}{t^{4}}} \\ 
        \Delta p_{t}(\vxi)
        &= \Ex\rs{\phi_{t}(\vxi - \vx)\cdot\frac{\norm{\vxi - \vx}_{2}^{2} - Dt^{2}}{t^{4}}}.
    \end{align}
\end{proposition}
\begin{proof} 
    我们使用\(p_{t}\)的卷积表示,即\eqref{eq:p_t_representation}。首先取时间导数,一个计算表明\Cref{prop:dutis}适用,\footnote{我们使用\(f_{t}(\vxi) = p(\vxi) \phi_{t}(\vxi - \vx)\),注意到它在\(\vxi\)上是二阶连续可微的,在\(t\)上是(超过)二阶连续可微的。然后为了检查\(f_{t}\)的局部可积性,我们计算\(\pdv{f_{t}}{t}(\vxi) = f_{t}(\vxi)\cdot \frac{1}{t^{3}}(\norm{\vxi - \vx}_{2}^{2} - Dt^{2})\),这很容易检查在\(\vxi\)和\(t \in [t_{\min}, t_{\max}]\)上是可积的,其中\(t_{\min} > 0\)。(确实,\(f_{t}\)有指数衰减的尾部,所以乘积中的二次项不成问题。)}所以我们可以将导数移到积分/期望内部:
    \begin{equation}
        \pdv{p_{t}}{t}(\vxi) = \pdv*{\Ex[\phi_{t}(\vxi - \vx)]}{t} = \Ex\rs{\pdv*{\phi_{t}(\vxi - \vx)}{t}} = \pdv{\phi_{t}}{t} * p.
    \end{equation}
    同时,根据卷积的性质(\Cref{prop:diff_convolution})并利用\(p\)是紧支撑的(\Cref{assumption:entropy_x_compact_support}),
    \begin{equation}
        p_{t} = \phi_{t} * p \implies \nabla p_{t} = \nabla \phi_{t} * p \implies \nabla^{2}p_{t} = \nabla^{2}\phi_{t} * p \implies \Delta p_{t} = \Delta \phi_{t} * p.
    \end{equation}
    剩下的计算遵循\Cref{prop:normal_derivatives}。
\end{proof}


\begin{proposition}\label{prop:normal_derivatives}
    对于\(t > 0\)和\(\vxi \in \R^{D}\),有
    \begin{align}
        \pdv*{\phi_{t}}{t}(\vxi)
        &= \phi_{t}(\vxi) \cdot \frac{\norm{\vxi}_{2}^{2} - Dt^{2}}{t^{3}} \\
        \nabla \phi_{t}(\vxi)
        &= -\phi_{t}(\vxi)\cdot\frac{\vxi}{t^{2}} \\ 
        \nabla^{2} \phi_{t}(\vxi)
        &= \phi_{t}(\vxi)\cdot\frac{\vxi\vxi^{\top} - t^{2}\vI}{t^{4}} \\
        \Delta \phi_{t}(\vxi)
        &= \phi_{t}(\vxi) \cdot \frac{\norm{\vxi}_{2}^{2} - Dt^{2}}{t^{4}}.
    \end{align}
\end{proposition}
\begin{proof}
    直接计算。
\end{proof}



\subsubsection{在积分号下求导}

在本附录中,我们多次在积分号下求导,了解何时可以这样做是很重要的。有两种在积分号下求导的情况:
\begin{enumerate}
    \item 对积分\(\int f_{t}(\vxi)\odif{\vxi}\)关于辅助参数\(t\)求导。
    \item 对卷积\((f * g)(\vxi) = \int f(\vxi)g(\vxi - \vu)\odif{u}\)关于变量\(\vxi\)求导。
\end{enumerate}

对于第一类,我们给出一个具体的结果,陈述时未加证明,但可归因于\href{https://planetmath.org/differentiationundertheintegralsign}{链接的来源},该来源将以下结果作为关于微分算子和缓增广义函数相互作用的一个更一般定理的特例推导出来,这远超出了本书的范围。一个完整的正式参考文献可以在\cite{jones1982theory}中找到。
\begin{proposition}[\cite{jones1982theory}, 第11.12节]\label{prop:dutis}
    设\(f \colon (0, T) \times \R^{D} \to \R\)满足:
    \begin{itemize}
        \item \(f\)是\((t, \vxi)\)的联合可测函数;
        \item 对于勒贝格几乎每个\(\vxi \in \R^{D}\),函数\(t \mapsto f_{t}(\vxi)\)是绝对连续的;
        \item \(\pdv{f_{t}}{t}\)是局部可积的,即对于每个\([t_{\min}, t_{\max}] \subseteq (0, T)\)有
        \begin{equation}
            \int_{t_{\min}}^{t_{\max}}\int_{\R^{D}}\abs*{\pdv{f_{t}}{t}(\vxi)}\odif{\vxi} < \infty.
        \end{equation}
    \end{itemize}
    那么\(t \mapsto \int_{\R^{D}}f_{t}(\vxi)\odif{\vxi}\)是\((0, T)\)上的一个绝对连续函数,其导数为
    \begin{equation}
        \odv*{\int_{\R^{D}}f_{t}(\vxi)\odif{\vxi}}{t} = \int_{\R^{D}}\pdv*{f_{t}}{t}(\vxi)\odif{\vxi},
    \end{equation}
    对于几乎每个\(t \in (0, T)\)都成立。
\end{proposition}

对于第二类,我们给出另一个具体的结果,陈述时未加证明,但在\cite{brezis2011functional}中有完整的形式化。
\begin{proposition}[\cite{brezis2011functional}, 命题4.20]\label{prop:diff_convolution}
    设\(f\)是具有紧支撑的\(k\)次连续可微函数,且\(g\)是局部可积的。那么由下式定义的卷积\(f * g\)
    \begin{equation}
        (f * g)(\vxi) := \int_{\R^{D}}f(\vu)g(\vxi - \vu)\odif{\vu}
    \end{equation}
    是\(k\)次连续可微的,其\(k\)阶导数为
    \begin{equation}
        \nabla^{k}(f * g) =(\nabla^{k}f) * g. 
    \end{equation}
\end{proposition}
虽然书中没有,但一个简单的分部积分论证表明,如果\(g\)也是\(k\)次可微的,那么我们可以“交换”正则性:
\begin{equation}
    \nabla^{k}(f * g) = f * (\nabla^{k} g).
\end{equation}

\end{document}
